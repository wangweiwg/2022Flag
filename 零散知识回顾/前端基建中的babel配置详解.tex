参考文章地址:https://juejin.cn/post/7025237833543581732

以webpack配置为例:
主要涉及到三个相关插件:
* babel-loader:识别配置文件和接受对应参数的函数,
* babel-core:babel编译过程中的核心库,将代码进行词法分析--语法分析--语义分析过程从而生成AST抽象语法树,
  从而对树操作后再通过编译成为新的代码。babel-core通过transform方法将我们的代码进行编译。
* babel-preset-env:告诉babel以什么样的规则进行代码转换

在 webpack 中 loader 的本质就是一个函数,接受我们的源代码作为入参同时返回新的内容


babel相关polyfill内容:
* 最新ES语法,比如:箭头函数、let/const
* 最新ES API,比如: Promise
* 最新ES实例/静态方法,比如:String.prototype.include
babel-preset-env 仅仅只会转化最新的es语法,并不会转化对应的API和实例方法,比如 ES 6中的Array.from静态方法。
babel是不会转译这个方法的,如果想在低版本浏览器中识别并且运行Array.form方法达到我们的预期就需要额外引入 polyfill
进行在Array上添加实例这个方法。其实可以稍微简单总结一下,语法层面的转化preset-env完全可以胜任。但是一些内置方法模块,
仅仅通过preset-env的语法转化是无法进行识别转化的,所以就需要一系列类似”垫片“的工具进行补充实现这部分内容的低版本代码
实现。这就是所谓的polyfill的作用。

针对于polyfill方法的内容,babel中涉及两个方面来解决:
@babel/polyfill:通过babelPolyfill通过往全局对象上添加属性以及直接修改内置对象的Prototype上添加方法实现polyfill。造成全局污染
@babel/runtime
@babel/plugin-transform-runtime

