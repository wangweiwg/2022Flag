Vue2.x与Vue3的对比

Vue2                                Vue3

1、生命周期
beforeCreate                        setup
created                             setup
beforeMount                         onBeforeMount
mounted                             onMounted
beforeUpdate                        onBeforeUpdate
updated                             onUpdated
beforeDestory                       onBeforeUnmount
destoryed                           onUnmounted
errCaptured                         onErrorCaptured

2.1、使用proxy 代替defineProperty
vue2对数组对象无法深层监听,因为每次渲染都是将data里的数据通过definProperty进行响应式或者双向绑定上,
之前没有后加的属性是不会被绑定上的,也就不会触发更新渲染
vue3使用proxy,不需要关心里面有什么属性,proxy有13种配置项,可以做到更细致,

2.2、两者兼容性
vue2.x之所以只能兼容到IE8就是因为defineProperty无法兼容IE8,其它浏览器也会存在轻微兼容问题
vue3中proxy的话除了IE,其它浏览器都兼容,说明vue3直接放弃了IE的兼容,个人感觉IE已经没人用了

2.3、Diff算法的提升
vue2.x以往的渲染策略是提供类似于HTML的模板语法,但是,它是将模板编译成渲染函数来返回虚拟DOM。Vue框架通过递归遍历
两个虚拟DOM,并比对每个节点上的每个属性,来确定实际DOM的哪些部分需要更新。

vue3中,编译器和运行时需要协同工作,大大提升了内存使用率减少垃圾回收、减少了DOM操作的开销、减少CPU占用率


2.4、typescript的支持
vue2中使用的都是js,它本身没有类型系统这个概念,typescript异常火爆,它的崛起是有原因的。因为对于规模很大的项目,没有类型声明,
后期维护和代码的阅读都是很头疼的事,所以程序猿还是很迫切的需要vue支持ts

2.4.1-如何实现
最终Vue借鉴了react hook实现了更自由的编程方式,提出了Composition API、Composition API不需要通过指定一长串选项来定义组件,
而是允许用户像写函数一样自由地表达,组合和重用有状态的组件逻辑,同时提出出色的ts支持


2.5、打包体积变化
vue2.x打包会随着依赖包和框架特性的增多,有时候不必要的,未使用的代码文件都被打包进去了,所以后期项目大了,打包文件会特别多还很大
vue3中,通过大多数全局API和内部程序移动到javascript 的module.exports属性上,这在现代模式下的module bundler能够静态地分析
模块依赖关系,并删除与未使用的module.exports 属性相关的代码。只有在模板中实际使用某一个特性时,该代码才导入该特性的帮助程序

2.6、其它API和功能的改动
vue-cli 从v4.5.0开始提供Vue3预设
vue-router 4.0开始提供了vue3支持
vuex 4.0提供vue3支持,API与vue2.x基本相同

2.7、组件API的使用
setup替代了以前的beforeCreate和created,类似于初始化的功能

2.8、ref、toRef、toRefs
ref: 就是简单的响应式变量
toRef: 就是把不是响应的变量变为响应变量
toRefs: 就是把reactive修饰的响应对象分解为单个的ref响应变量

2.9、watch、watchEffect
watch第一个参数是被监听的变量,第二个是执行的回调函数
watchEffect里面的变量发生变化都会执行,并且第一次渲染会立即执行,没有变化前后的参数,无法监听整个reactive
watch可以直接监听ref和reactive绑定的对象,watchEffect不可以,(ref的值要.value,reactive的值要具体到
内部属性),只会执行第一次

3.0、函数组件,
所有函数都有props、context参数

3.1、vue3对打包体积做了深入优化,体积比原来小了一半,vue3不需要diff递归,理论上更快

落后就会被淘汰,升级迭代是家常便饭,无须抵触新技术,最好的方式就是拥抱新技术,只有这样
才不至于在前端的浪潮中被淘汰
